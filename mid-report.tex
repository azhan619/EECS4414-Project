%%
%% This is file `sample-authordraft.tex',
%% generated with the docstrip utility.
%%
%% The original source files were:
%%
%% samples.dtx  (with options: `authordraft')
%% 
%% IMPORTANT NOTICE:
%% 
%% For the copyright see the source file.
%% 
%% Any modified versions of this file must be renamed
%% with new filenames distinct from sample-authordraft.tex.
%% 
%% For distribution of the original source see the terms
%% for copying and modification in the file samples.dtx.
%% 
%% This generated file may be distributed as long as the
%% original source files, as listed above, are part of the
%% same distribution. (The sources need not necessarily be
%% in the same archive or directory.)
%%
%% The first command in your LaTeX source must be the \documentclass command.
\documentclass[sigconf,nonacm]{acmart}
%% NOTE that a single column version may required for 
%% submission and peer review. This can be done by changing
%% the \doucmentclass[...]{acmart} in this template to 
%% \documentclass[manuscript,screen]{acmart}
%% 
%% To ensure 100% compatibility, please check the white list of
%% approved LaTeX packages to be used with the Master Article Template at
%% https://www.acm.org/publications/taps/whitelist-of-latex-packages 
%% before creating your document. The white list page provides 
%% information on how to submit additional LaTeX packages for 
%% review and adoption.
%% Fonts used in the template cannot be substituted; margin 
%% adjustments are not allowed.

%%
%% \BibTeX command to typeset BibTeX logo in the docs
\AtBeginDocument{%
  \providecommand\BibTeX{{%
    \normalfont B\kern-0.5em{\scshape i\kern-0.25em b}\kern-0.8em\TeX}}}

%% Rights management information.  This information is sent to you
%% when you complete the rights form.  These commands have SAMPLE
%% values in them; it is your responsibility as an author to replace
%% the commands and values with those provided to you when you
%% complete the rights form.

%% These commands are for a PROCEEDINGS abstract or paper.



%%
%% Submission ID.
%% Use this when submitting an article to a sponsored event. You'll
%% receive a unique submission ID from the organizers
%% of the event, and this ID should be used as the parameter to this command.
%%\acmSubmissionID{123-A56-BU3}

%%
%% The majority of ACM publications use numbered citations and
%% references.  The command \citestyle{authoryear} switches to the
%% "author year" style.
%%
%% If you are preparing content for an event
%% sponsored by ACM SIGGRAPH, you must use the "author year" style of
%% citations and references.
%% Uncommenting
%% the next command will enable that style.
%%\citestyle{acmauthoryear}

%%
%% end of the preamble, start of the body of the document source.

\begin{document}


%%
%% The "title" command has an optional parameter,
%% allowing the author to define a "short title" to be used in page headers.
\title{EECS4414: Network Analysis on Flight Delays and Prediction}

%%
%% The "author" command and its associated commands are used to define
%% the authors and their affiliations.
%% Of note is the shared affiliation of the first two authors, and the
%% "authornote" and "authornotemark" commands
%% used to denote shared contribution to the research.

\author{Azhan Khan Ghori}
\affiliation{%
  \institution{York University}
  \city{Toronto}
  \country{Canada}}
\email{azhan619@my.yorku.ca}

\author{Ramia Ejaz}
\affiliation{%
  \institution{York University}
  \city{Toronto}
  \country{Canada}}
\email{ramia30@my.yorku.ca}


\author{Maira Afzaal}
\affiliation{%
  \institution{York University}
  \city{Toronto}
  \country{Canada}}
\email{afzaal@my.yorku.ca}

\author{Nicholas Cerisano}
\affiliation{%
  \institution{York University}
  \city{Toronto}
  \country{Canada}}
\email{nichceri@my.yorku.ca}



%%
%% By default, the full list of authors will be used in the page
%% headers. Often, this list is too long, and will overlap
%% other information printed in the page headers. This command allows
%% the author to define a more concise list
%% of authors' names for this purpose.


%%
%% The abstract is a short summary of the work to be presented in the
%% article.


%%
%% The code below is generated by the tool at http://dl.acm.org/ccs.cfm.
%% Please copy and paste the code instead of the example below.
%%


%%
%% Keywords. The author(s) should pick words that accurately describe
%% the work being presented. Separate the keywords with commas.
%\keywords{datasets, neural networks, gaze detection, text tagging}

%% A "teaser" image appears between the author and affiliation
%% information and the body of the document, and typically spans the
%% page.


%%
%% This command processes the author and affiliation and title
%% information and builds the first part of the formatted document.
\maketitle

\section{Motivation and Problem Definition}

\subsection{Motivation}
Anyone who has travelled on an airplane has most likely experienced flight delays. Some of the reasons contributing to delay in flights are the following: security, weather conditions, shortage of parts, technical and airplane equipment issues, and flight crew delays; regardless of the reason for the delay, they cause inconvenience for travellers and financial loss for the airlines. Flight delays are unavoidable and have a negative economic effect on passengers, airlines and airports. 


Flight delay prediction is critical because it can increase customer satisfaction and the income of airline agencies. The airline is an important factor for travellers to consider when they decide to travel. According to the 2015 Kaggle dataset from the U.S Department of Transportation, the following airlines had the highest delay times: Envoy Air (MQ), JetBlue (B6), United Airlines (UA), Frontier Airlines (F9), and Spirit Airlines (NK). However, the travellers might need to consider airports they tend to fly to and from as airports play a significant role in flight delays. Among the five, Chicago O'Hare International Airport (ORD) had the longest delays, with an average departure delay of 12.5 minutes and an average arrival delay of 22.5 minutes. The weather has been one of the biggest factors causing flight delays. However, the number of flights and connections going through an airport is more strongly associated with flight delays.  


In this project we are going to focus on flight delays prediction. There is an increasing need for building an accurate flight delay prediction model to minimize the financial loss and inconvenience caused by it. 

\subsection{Problem Definition}
Given data on flights and airports, we will construct a network that can be used to make analysis, predictions and conclusions on flight delays. We will look at various factors that impact flight delays. Airports are affected by other airports, the closer the relationship between airports, the more significant the impact\cite{Zeng2021}. Thus, we will study the problem of flight delay prediction from the perspective of the airport network.


\subsection{Related Studies}
In a study Wang, et. al.\cite{Wang2020}, a quantitative method has been used to compare network delays of the USA and China. By using the actual operational data, a method was proposed to construct an airport network containing delay information. The structural properties and betweenness centrality in the flight delay network were compared. By calculating the correlation of flight delay between airports, it is found that geographical location has a huge impact on flight delays. Also, the flight delay time series data were analyzed using the MF-DFA method. It is determined that flight delays present fractal characteristics that are hard to be described by a single fractal method. 


In the study of Zeng, et. al.\cite{Zeng2021}, a deep DGLSTM delay prediction model was proposed to study flight delay prediction through the airport network approach. In a graph network, airports are represented as nodes and for adjacent airports weights of edges are measured by the spherical distance between the two. Another LSTM model was used to mine time-domain features including encoding and decoding. The result of the study indicated that the methods present in this have higher accuracy and robustness. The method proposed in the paper was found to be better than the current methods. 


\section{Methodology}
\subsection{Network Construction}
The idea behind representing the flight data would be to have each node represent the airport while the edges will represent the flight route between the airports. The edges will be directed , that is if a node A has an edge connecting to node B then, we know there exists a flight between those two nodes ( airports ) . On the main network that is used for analysis, the weight will be the count of flights between those two in the year. Later on when working with models, the weights will adjust accordingly and will be changed. 

The network contains 322 nodes and around 4,600 edges, which means that the US Domestic flight network contains 322 airports and has over 4,600 different routes. The data used for this network construction was of 2015, hence there are a very small number of airports missing that were constructed or inaugurated after 2015, we plan to update this network later on with the final report, once we find all the correct data for those new airports. To construct the graph, we used the Networkx library from python and the dataset was downloaded from Kaggle. The files were downloaded in excel csv format, so we used Pandas framework to read and prepare the data for the graph. The dataset had an ORIGIN and DESTINATION column which has the airport’s official IATA code, firstly the data was grouped together using origin and destination airports and the additional column of count was added to keep track of number of flights between those routes. Then these 3 columns were given to Networkx to construct the graph.
\hfill \break

\textbf{2.1.2 Network Visualization} 

\hfill \break

Once the graph was created, we had to figure out a nice way to represent it visually so it is easier to understand the network topology. To plot the graph, first the Matplotlib library was used. Since the number of nodes and edges are very high in number, the result of the plot was not enough, hence we used the template provided by e.t Nguyen[1]. In this model, we used the Basemap package along with Matplotlib in Python. First a map for the US was imported using basemap and then other details were filled in like state borders, country and sea borders to make it look more understandable. Once the map was finalized, then we used our dataset to get the position of each airport in the form of longitude and latitude. Then using the position, nodes were plotted in the exact area with edges between them. Furthermore, to make the plot more insightful, we plotted the nodes in varying size, such that the airports with the most number of routes were bigger in size and then sizes were adjusted according to the number of routes for each airport. Then, the nodes with greater than 100 routes were highlighted in red color to make the most busiest airports stand-out in the network diagram, which is posted below. Figure 1.


\subsection{Centrality Measure}
Degree centrality focuses on connected nodes between entities. The more the neighbors are directly connected to each node, the greater the centrality. A high centrality measure means that the node has multiple edges directed to it. In the study of flight networks, if a node has high centrality it means that the airport has multiple routes in operation with other airports. Similarly, betweenness centrality is a measure of centrality in a graph based on shortest paths.It represents the degree of which nodes stand between each other. High betweenness corresponds to the degree of mediation role in the flight network and is a well known algorithm to identify and rank busiest hubs in the flight networks. Furthermore, we can make a list of airports with longest delays using the dataset and compare the results from the algorithm.\cite{datares_2020_analyzing}

\subsection{Predicting and Analyzing Flight Delays}
Building prediction models for flight delays has been studied extensively in different fields. One of the studies is based on Deep Graph-Embedded Neural networks. It uses the flight network to construct airport relationships, and the concept of spherical distance to weigh the edge between two different nodes. It is then integrated into the sequence-to-sequence neural network to construct a deep learning framework for delay predictions. Additionally, the other methods to analyze flight delays, is to introduce multidisciplinary frameworks of networks and control theory. It involves the study of real flight delay data, building a delay network and then analyzing the network to formalize their dynamics. Then this formalization is used to design optimal control of flight networks. The latter approach was recently proposed by Xiang et al.\cite{Niu2021}, and they claim that applying these methods has helped to reduce delay times and benefited airlines and passengers.

\subsection{Time-Series Correlation Analysis}

Correlation coefficients quantify the association between features of a dataset, or the network graph. The same method was discussed by Want et al.\cite{Wang2020}, where Pearson correlation coefficient was used to capture the correlation between flight delays in different airports. To conduct this analysis, a time-series data will be used with appropriate sampling rate to calculate the departure delays at each airport. The results from this measure can be used to check if the correlated airport has any similar characteristics that have an impact on flight delays.





\section{Evaluation}
\subsection{Data}
The data used will be provided by an open data set found on Kaggle
(https://www.kaggle.com/yuanyuwendymu/airline-delay-and-cancellation-data-2009-2018?select=2011.cs) and BTS (https://www.bts.gov/) detailing flight
delays and cancellations between 2009 and 2018. This dataset was chosen due to it having a lot of specific data related to individual flight delays
including all parts of the delay such as the carrier, date, and most importantly origin and destination. Network nodes will represent airports while
edges will represent the flights from origin to destination.
\subsection{Experiments}
We will analyze betweenness centrality in the constructed flight delay network. The analysis will contain visual elements such as data plots to better visualize and understand correlations in our data.

  
  

  
  











%%
%% The acknowledgments section is defined using the "acks" environment
%% (and NOT an unnumbered section). This ensures the proper
%% identification of the section in the article metadata, and the
%% consistent spelling of the heading.


%%
%% The next two lines define the bibliography style to be used, and
%% the bibliography file.
\bibliographystyle{ACM-Reference-Format}
\bibliography{sample-base}

%%
%% If your work has an appendix, this is the place to put it.
\appendix


\end{document}
\endinput
%%
%% End of file `sample-authordraft.tex'.
